\documentclass[fleqn]{article}
\usepackage[margin=1in]{geometry}
\usepackage[nodisplayskipstretch]{setspace}
\usepackage{amsmath, nccmath, bm}
\usepackage{amssymb}
\usepackage{enumitem}
\usepackage{graphicx}
\usepackage{float}
\usepackage{listings}
\usepackage{hyperref}
\usepackage[svgnames]{xcolor}
\graphicspath{{./images}}

\hypersetup{
    colorlinks=true,
    linkcolor=black,
    filecolor=black,      
    urlcolor=blue
    }

\newcommand{\zerodisplayskip}{
	\setlength{\abovedisplayskip}{0pt}%
	\setlength{\belowdisplayskip}{0pt}%
	\setlength{\abovedisplayshortskip}{0pt}%
	\setlength{\belowdisplayshortskip}{0pt}%
	\setlength{\mathindent}{0pt}}
	
\definecolor{vgreen}{RGB}{104,180,104}
\definecolor{vblue}{RGB}{49,49,255}
\definecolor{vorange}{RGB}{255,143,102}

\lstdefinestyle{verilog-style}
{
    language=Verilog,
    basicstyle=\small\ttfamily,
    keywordstyle=\color{vblue},
    identifierstyle=\color{black},
    commentstyle=\color{vgreen},
    numbers=left,
    numberstyle=\tiny\color{black},
    numbersep=10pt,
    tabsize=8,
    moredelim=*[s][\colorIndex]{[}{]},
    literate=*{:}{:}1
}

\lstset{style={verilog-style},showstringspaces=false}

\makeatletter
\newcommand*\@lbracket{[}
\newcommand*\@rbracket{]}
\newcommand*\@colon{:}
\newcommand*\colorIndex{%
    \edef\@temp{\the\lst@token}%
    \ifx\@temp\@lbracket \color{black}%
    \else\ifx\@temp\@rbracket \color{black}%
    \else\ifx\@temp\@colon \color{black}%
    \else \color{vorange}%
    \fi\fi\fi
}
\makeatother

\newcommand{\code}[1]{%
	\colorbox{Gainsboro}{\texttt{#1}}%
}

\title{Homework 4}
\author{Owen Sowatzke}
\date{April 16, 2025}

\begin{document}

	\offinterlineskip
	\setlength{\lineskip}{12pt}
	\zerodisplayskip
	\maketitle
	
	\begin{enumerate}
		\item ~
		
			\begin{figure}[H]
				\centerline{\includegraphics[width=0.5\textwidth]{circuit_question1.png}}
				\label{fig::circuit_question1}
			\end{figure}

			\begin{enumerate}
			\item[1.] Write out the output function in terms of A and B.			
			
			\begin{equation*}
				\mathbf{Out = \overline{AB}}
			\end{equation*}
			
			\item[2.] $V_{DD}=2.0\text{V}$, $|V_{th}|=0.6\text{V}$ for all PMOS and NMOS transistors. $C_L/C_A=3$, $C_L/C_B=5$; What is the voltage value at X and at output nodes ($V_x$ = ? and $V_{out}$ = ?)
			
			$V_{\text{out}}$ will be charged to $V_{DD}$ during the charge phase. Assume that all other nodes are charged to $0\text{V}$.
			
			When $A$ rises from 0 to 1. $C_L$ will discharge through $M_a$ and share charge with $C_A$. Because $B=0$, $M_b$ should be in the cutoff state, so there will be no charge sharing with $C_B$.
			
			Start by assuming that there is no voltage drop across $V_a$.
			
			\begin{equation*}
				C_LV_{DD} = (C_L + C_A)V_{out} 
			\end{equation*}			
			
			\begin{equation*}
				\Rightarrow V_x = V_{out} = \frac{C_LV_{DD}}{C_L + C_A} = \frac{3C_AV_{DD}}{3C_A + C_A} = \frac{3V_{DD}}{4} = 1.5\text{V}
			\end{equation*}
			
			 However, for $M_a$ to be on $V_{gs} < V_{th} \Rightarrow V_x \leq 1.4\text{V}$.
			
			Therefore, $V_x$ will charge to 1.4V, which will consume some of the charge in $C_L$.
			
			$Q_x = 1.4C_A$
			
			$\Rightarrow Q_{out} = C_LV_DD - Q_x = 2C_L - 1.4C_A = 6C_A - 1.4C_A = 4.6C_A$
			
			$\therefore V_\text{out} = Q_{out}/C_L = 4.6C_A/3C_A = \mathbf{1.53\text{\textbf{V}}},\ V_x = \mathbf{1.4\text{\textbf{V}}}$
			
			\end{enumerate}
		
		\item Use static complementary CMOS design, Pseudo-NMOS, domino dynamic design to implement
		
		$\mathbf{Y = \overline{(A + B)C + D}}$
		
		Assume we use short channel transistors. We use 0.10 um process with 1GHz clock frequency and 1.0V power supply with 121mA/mm. We would like to keep the power density the same, but we would like to scale current density to $81\text{mA}/\text{mm}^2$.
		
		\item In the following figure, what king of logic is implemented in this circuit? Write out the Z function.
		
			\begin{figure}[H]
				\centerline{\includegraphics[width=0.5\textwidth]{circuit_question3.png}}
				\label{fig::circuit_question3}
			\end{figure}

			The dynamic logic gates implement the following equations:
			
			$W = \overline{AB}$
			
			$Y = \overline{XC}$
			
			The static CMOS gates implement the following equations:
			
			$X = \bar{W}$
			
			$Z = \bar{Y}$
			
			$\therefore$ Z can be computed as follows:
			
			$Z = \overline{\overline{XC}} = XC = \bar{W}C = \overline{\overline{AB}}C = \mathbf{ABC}$
			
			Alternatively, we can recognize Z as two chained domino AND gates.
			
		\item ~
		
			\begin{enumerate}
			
			\item[1.] What is the size of each transistor for $Y = \overline{AB + C}$ if we use static complementary design?
			
			In the static complementary design, we size the transistors to get the same drive as a unit inverter. For series connections, this means that the width of the transistors need to be doubled to get the same equivalent resistance. For the parallel connections, we consider only the case when one branch is on and don't increase the width of the transistor. In Figure \ref{fig::circuit4a}, we show the static complementary design and the sizes of each transistors.
			
			\begin{figure}[H]
				\centerline{\includegraphics[width=0.25\textwidth]{circuit4a.png}}
				\caption{Static Complementary Design for $Y = \overline{AB + C}$.}
				\label{fig::circuit4a}
			\end{figure}
			
			\item[2.] What type of CMOS design is the depicted in the figure below? Describe its functionality and write down its logic equation?
			
			\begin{figure}[H]
				\centerline{\includegraphics[width=0.5\textwidth]{circuit4b.png}}
				\label{fig::circuit4b}
			\end{figure}
			
			The design illustrates differential cascode voltage switch logic (DCVSL). When A=1 and B=0, $\overline{\text{Out}}$ is connected to ground, and the PMOS transistor on the left is turned on pulling $\text{Out}$ to $V_{DD}$. Similarly, when A=0 and B=1, $\overline{\text{Out}}$ is connected to ground, and the PMOS transistor on the left is turned on pulling $\text{Out}$ to $V_{DD}$. Next, if A=B=0, $\text{Out}$ is connected to ground and the PMOS transistor on the right is turned on pulling $\overline{\text{Out}}$ to $V_{DD}$. Finally, if A=B=1, $\text{Out}$ is connected to ground and the PMOS transistor on the right is turned on pulling $\overline{\text{Out}}$ to $V_{DD}$. Our circuit implements the following logic equations:
			
			\begin{equation*}
				\text{Out} = \bar{A}B + A\bar{B}
			\end{equation*}
			
			\begin{equation*}
				\overline{\text{Out}} = \overline{\bar{A}B + A\bar{B}} = (\overline{\bar{A}B})(\overline{A\bar{B}}) = (A + \bar{B})(\bar{A} + B) = A\bar{A} + \bar{A}\bar{B} + AB + B\bar{B} = AB + \bar{A}\bar{B}
			\end{equation*}
			
			We recognize the equation for $\text{Out}$ as XOR, and the equation for $\overline{\text{Out}}$ as XNOR, where $\text{Out}$ is the XOR output. Therefore, our gate implements XOR/XNOR.
			
			\end{enumerate}			 
	\end{enumerate}

\end{document}