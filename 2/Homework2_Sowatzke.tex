\documentclass[fleqn]{article}
\usepackage[margin=1in]{geometry}
\usepackage[nodisplayskipstretch]{setspace}
\usepackage{amsmath, nccmath, bm}
\usepackage{amssymb}
\usepackage{enumitem}
\usepackage{graphicx}
\usepackage{float}
\usepackage{listings}
\usepackage{hyperref}
\usepackage[svgnames]{xcolor}
\graphicspath{{./images}}

\hypersetup{
    colorlinks=true,
    linkcolor=black,
    filecolor=black,      
    urlcolor=blue
    }

\newcommand{\zerodisplayskip}{
	\setlength{\abovedisplayskip}{0pt}%
	\setlength{\belowdisplayskip}{0pt}%
	\setlength{\abovedisplayshortskip}{0pt}%
	\setlength{\belowdisplayshortskip}{0pt}%
	\setlength{\mathindent}{0pt}}
	
\definecolor{vgreen}{RGB}{104,180,104}
\definecolor{vblue}{RGB}{49,49,255}
\definecolor{vorange}{RGB}{255,143,102}

\lstdefinestyle{verilog-style}
{
    language=Verilog,
    basicstyle=\small\ttfamily,
    keywordstyle=\color{vblue},
    identifierstyle=\color{black},
    commentstyle=\color{vgreen},
    numbers=left,
    numberstyle=\tiny\color{black},
    numbersep=10pt,
    tabsize=8,
    moredelim=*[s][\colorIndex]{[}{]},
    literate=*{:}{:}1
}

\lstset{style={verilog-style},showstringspaces=false}

\makeatletter
\newcommand*\@lbracket{[}
\newcommand*\@rbracket{]}
\newcommand*\@colon{:}
\newcommand*\colorIndex{%
    \edef\@temp{\the\lst@token}%
    \ifx\@temp\@lbracket \color{black}%
    \else\ifx\@temp\@rbracket \color{black}%
    \else\ifx\@temp\@colon \color{black}%
    \else \color{vorange}%
    \fi\fi\fi
}
\makeatother

\newcommand{\code}[1]{%
	\colorbox{Gainsboro}{\texttt{#1}}%
}

\title{Homework 2}
\author{Owen Sowatzke}
\date{February 24, 2025}

\begin{document}

	\offinterlineskip
	\setlength{\lineskip}{12pt}
	\zerodisplayskip
	\maketitle
	
	\begin{enumerate}
		\item Based on the figures below mention which figure depicts the mode of operation and briefly describe its operation in terms of the figure.
		
		\begin{enumerate}
		
			\item Figure \ref{fig::mode_of_operations_part_a} depicts the cutoff region. In this region, there is no channel between the source and drain. The n+ doping around the source and drain forms a reversed-biased PN junctions with the p-type body, prevent current from flowing. As such, the current between the source and drain ($I_{D}$) is approximately zero.
			
			\begin{figure}[H]				
				\centerline{\includegraphics[width=0.5\textwidth]{mode_of_operations_part_a.png}}
				\caption{CMOS Transistor in Cutoff Region}
				\label{fig::mode_of_operations_part_a}
			\end{figure}
		
			\item Figure \ref{fig::mode_of_operations_part_b} depicts the resistive region. In this region, the positive gate charge attracts negative charge from the p-type body, forming a n-channel. Between, the p and n-type regions, the drift and diffusion currents counteract to form a depletion zone. Current cannot flow through the n-channel without a potential difference between drain and source. When this potential exists, electrons flow from source to drain, creating a current from the drain to the source. For small potential difference (small values of $V_D$), the current grows linearly with applied voltage, hence the naming of this region.
			
			% The diffusion current causes negative charge to form in the p-type body and positive charge to form in the n-type body. This eventually leads to an electric field and in turn a drift current, which prevents the movement of additional charged.
			
			\begin{figure}[H]				
				\centerline{\includegraphics[width=0.5\textwidth]{mode_of_operations_part_b.png}}
				\caption{CMOS Transistor in Resistive Region}
				\label{fig::mode_of_operations_part_b}
			\end{figure}
			
			\item Figure \ref{fig::mode_of_operations_part_c} depicts the start of the saturation region. Here the channel is being pinched off at the drain because the gate to drain voltage ($V_{gd}$) falls below the threshold voltage ($V_t$). This causes the region of the channel near the drain to no longer be inverted. However, the potential difference still accelerates the electrons from source to drain. As we further increase $V_{ds}$, the drain to source current $I_{D}$ saturates at a value of $I_{D_{sat}}$.
			
			\begin{figure}[H]				
				\centerline{\includegraphics[width=0.5\textwidth]{mode_of_operations_part_c.png}}
				\caption{CMOS Transistor in Saturation Region }
				\label{fig::mode_of_operations_part_c}
			\end{figure}
		
			\item Figure \ref{fig::mode_of_operations_part_d} depicts the saturation region.
			
			\begin{figure}[H]				
				\centerline{\includegraphics[width=0.5\textwidth]{mode_of_operations_part_d.png}}
				\caption{CMOS Transistor in Saturation Region}
				\label{fig::mode_of_operations_part_d}
			\end{figure}
			
		\end{enumerate}
		
		\item Describe the differences in the I-V characteristics of NMOS and PMOS transistors. How does mobility ($\mu$) affect the drain current ($I_D$) in both cases? Why must PMOS transistors be wider than NMOS to provide the same current? Support your answer with relevant equations.
		
		\item Using the I-V characteristics of an NMOS transistor, explain the different regions of operation (Cutoff, Linear, and Saturation). Derive the drain current ($I_D$) equations for each region and discuss how the gate-source voltage ($V_{gs}$) and drain-source voltage ($V_{ds}$) determine the mode of operation. 

		\item For an NMOS transistor, the drain current in the saturation region is given by:
		
		\begin{equation*}
			I_D = \frac{1}{2}{\mu_n}{C_{ox}}\frac{W}{L}(V_{GS}-V_{T})^2
		\end{equation*}
		
		where $C_{ox} = \frac{\varepsilon_{ox}}{t_{ox}}$ is the oxide capacitance per unit area.
		
		\begin{enumerate}
			\item If $\mu_n = 450 \frac{\text{cm}^2}{V}$, $C_{ox} = 10^{-2}\frac{\text{F}}{\text{m}^2}$, $\frac{W}{L} = 10$, $V_{GS} = 2\ \text{V}$, and $V_T = 0.7\ \text{V}$, compute $I_D$.
			
			\begin{equation*}
				I_D = \frac{1}{2}\left(450\frac{\text{cm}^2}{\text{V}}\right)\left(\frac{\text{m}}{100\text{cm}}\right)^2\left(10^{-2}\frac{\text{F}}{\text{m}^2}\right)(10)\left(2\text{V} - 0.7\text{V}\right)^2 = \mathbf{0.0380 \textbf{\text{A}}}
			\end{equation*}
			
			\item How does $I_D$ change if $\frac{W}{L}$ is doubled?
			
			$I_D$ doubles when $\frac{W}{L}$ is doubled.
			
		\end{enumerate}
		
		\item The die yield is given by the formula:
		
		\begin{equation*}
			\text{yield} = \frac{1}{(1+\text{Defects per unit area} \cdot \text{Die area})^{\alpha}}
		\end{equation*}
		
		where $\alpha$ is a process-dependent factor, typically around 3.
		
		\begin{enumerate}
			\item If a wafer has a defect density of $0.5\ \text{defects per cm}^2$, and the die area is $0.2\ \text{cm}^2$, calculate the expected yield.
			
			Assume that $\alpha$ is 3.
			
			\begin{equation*}
				\Rightarrow \text{yield} = \frac{1}{[1 + (0.5\ \text{defects per cm}^2)(0.2\ \text{cm}^2)]^3} = \mathbf{75.13\%}
			\end{equation*}
			
			\item If the defect density increases to $1.0\ \text{defects per cm}^2$, what is the new yield?
			
			Assume that $\alpha$ is 3.
			
			\begin{equation*}
				\Rightarrow \text{yield} = \frac{1}{[1 + (1.0\ \text{defects per cm}^2)(0.2\ \text{cm}^2)]^3} = \mathbf{57.87\%}
			\end{equation*}
			
		\end{enumerate}
	\end{enumerate}

\end{document}