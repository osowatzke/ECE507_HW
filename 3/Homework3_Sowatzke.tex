\documentclass[fleqn]{article}
\usepackage[margin=1in]{geometry}
\usepackage[nodisplayskipstretch]{setspace}
\usepackage{amsmath, nccmath, bm}
\usepackage{amssymb}
\usepackage{enumitem}
\usepackage{graphicx}
\usepackage{float}
\usepackage{listings}
\usepackage{hyperref}
\usepackage[svgnames]{xcolor}
\graphicspath{{./images}}

\hypersetup{
    colorlinks=true,
    linkcolor=black,
    filecolor=black,      
    urlcolor=blue
    }

\newcommand{\zerodisplayskip}{
	\setlength{\abovedisplayskip}{0pt}%
	\setlength{\belowdisplayskip}{0pt}%
	\setlength{\abovedisplayshortskip}{0pt}%
	\setlength{\belowdisplayshortskip}{0pt}%
	\setlength{\mathindent}{0pt}}
	
\definecolor{vgreen}{RGB}{104,180,104}
\definecolor{vblue}{RGB}{49,49,255}
\definecolor{vorange}{RGB}{255,143,102}

\lstdefinestyle{verilog-style}
{
    language=Verilog,
    basicstyle=\small\ttfamily,
    keywordstyle=\color{vblue},
    identifierstyle=\color{black},
    commentstyle=\color{vgreen},
    numbers=left,
    numberstyle=\tiny\color{black},
    numbersep=10pt,
    tabsize=8,
    moredelim=*[s][\colorIndex]{[}{]},
    literate=*{:}{:}1
}

\lstset{style={verilog-style},showstringspaces=false}

\makeatletter
\newcommand*\@lbracket{[}
\newcommand*\@rbracket{]}
\newcommand*\@colon{:}
\newcommand*\colorIndex{%
    \edef\@temp{\the\lst@token}%
    \ifx\@temp\@lbracket \color{black}%
    \else\ifx\@temp\@rbracket \color{black}%
    \else\ifx\@temp\@colon \color{black}%
    \else \color{vorange}%
    \fi\fi\fi
}
\makeatother

\newcommand{\code}[1]{%
	\colorbox{Gainsboro}{\texttt{#1}}%
}

\title{Homework 3}
\author{Owen Sowatzke}
\date{April 2, 2025}

\begin{document}

	\offinterlineskip
	\setlength{\lineskip}{12pt}
	\zerodisplayskip
	\maketitle
	
	\begin{enumerate}
		\item A digital system-on-chip (SoC) is fabricated using a \textbf{1V, 65nm} process, where the drawn channel lengths are \textbf{50nm}, and $\mathbf{\lambda}$ \textbf{= 25nm}. The chip consists of \textbf{600 million transistors}, out of which \textbf{70 million} are used for logic, while the remaining transistors are embedded in memory arrays. 
		
		Key parameters for transistor width: 

		\begin{itemize}
			\item \textbf{Logic transistors:} Average width of $\mathbf{14\lambda}$	 
			\item \textbf{Memory transistors:} Average width of $\mathbf{5\lambda}$
		\end{itemize}
			 
		Memory arrays are structured into banks, and only the necessary bank is activated during operation, with a \textbf{memory activity factor of 0.03}. The static CMOS logic gates have an \textbf{average switching activity factor of 0.09}. Each transistor contributes $\mathbf{1.2\ \text{\textbf{fF}}/\text{\textbf{R}}_\text{\textbf{m}}}$ of gate capacitance and $\mathbf{1.0\ \text{\textbf{fF}}/\text{\textbf{R}}_\text{\textbf{m}}}$ of diffusion capacitance. Use the given conversion factor of $\mathbf{0.025\ \text{\textbf{R}}_\text{\textbf{m}}}$ \textbf{per} $\mathbf{\lambda}$ \textbf{for both logic and memory. Ignore wiring capacitance.}
		
		Calculate the \textbf{switching power consumption} of the chip when operating at \textbf{800 MHz}. Show all steps and formulas used in your calculations.
		
		\begin{equation*}
			C = 1.2\ \text{fF}/\text{R}_\text{m}\ \text{(gate)}\ + 1.0\ \text{fF}/\text{R}_\text{m}\ \text{(diffusion)} = 2.2\ \text{fF}/\text{R}_\text{m}
		\end{equation*}
		
		\begin{equation*}
			C_{logic} = (70 \times 10^6)(14\lambda)(0.025\ \text{R}_\text{m}/\lambda)(2.2\ \text{fF}/\text{R}_\text{m}) = 53.9\ \text{nF}
		\end{equation*}
		
		\begin{equation*}
			C_{mem} = (530 \times 10^6)(5\lambda)(0.025\ \text{R}_\text{m}/\lambda)(2.2\ \text{fF}/\text{R}_\text{m}) = 145.75\ \text{nF}
		\end{equation*}
		
		\begin{equation*}
			P_{switching} = {\alpha}CV_{DD}^2f = ({\alpha_{mem}}C_{mem} + {\alpha_{logic}}C_{logic})V_{DD}^2f
		\end{equation*}
		
		\begin{equation*}
			= \left[0.03(145.75\ \text{nF}) + 0.09(53.9\ \text{nF})\right](1^2)(800 \times 10^6) = \mathbf{7.3788}\ \text{\textbf{W}}
		\end{equation*}
		
		\item Using the same system-on-chip described in Question 1, assume the following leakage parameters:
		
		\begin{itemize}
			\item \textbf{Subthreshold leakage:}
				\begin{itemize}[label={--}]
					\item Low-threshold (high-leakage) devices: $100\ \text{nA}/\text{R}_\text{m}$
					\item High-threshold (low-leakage) devices: $10\ \text{nA}/\text{R}_\text{m}$
				\end{itemize}
			\item \textbf{Gate leakage:} $5\ \text{nA}/\text{R}_\text{m}$
			\item \textbf{Junction leakage:} Negligible.
		\end{itemize}
		
		For leakage, the memory arrays are designed with high-threshold (i.e. low-leakage) devices, while the logic uses high-threshold devices only on the 8\% of the paths that are most performance critical; the remaining 92\% of the logic uses low-threshold (high-leakage) devices. Assume that (on average) half the transistors are OFF and contribute subthreshold leakage and half are ON and contribute gate leakage.

		\textbf{Estimate the static power consumption.}
		
		\begin{equation*}
			W_{\text{normal-V}_\text{t}} = (70 \times 10^6)(14\lambda)(0.025\ \text{R}_\text{m}/\lambda)(0.08) = 1.96 \times 10^6\ \text{R}_\text{m}
		\end{equation*}
		
		\begin{equation*}
			W_{\text{high-V}_\text{t}} = \left[(70 \times 10^6)(14\lambda)(0.92) + (530 \times 10^6)(5\lambda)\right](0.025\ \text{R}_\text{m}/\lambda) = 88.79 \times 10^6\ \text{R}_\text{m}
		\end{equation*}
		
		\begin{equation*}
			I_\text{sub} = \left[W_{\text{normal-V}_\text{t}} \times 100\ \text{nA}/\text{R}_\text{m} + W_{\text{high-V}_\text{t}} \times 10\ \text{nA}/\text{R}_\text{m}\right]/2 = 541.95\ \text{mA}
		\end{equation*}			
			
		\begin{equation*}
			I_\text{static} = \left[(W_{\text{normal-V}_\text{t}} + W_{\text{high-V}_\text{t}}) \times 5\ \text{nA}/\text{R}_\text{m}\right]/2 = 226.875\ \text{mA}
		\end{equation*}	
		
		\begin{equation*}
			P_\text{static} = (541.95\ \text{mA} + 226.875\ \text{mA})(1.0 V) = \mathbf{768.825}\ \text{\textbf{mW}}
		\end{equation*}	
	\end{enumerate}

\end{document}